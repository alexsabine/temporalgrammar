\documentclass[11pt,a4paper]{article}

% Packages
\usepackage{amsmath,amsthm,amssymb,amsfonts}
\usepackage{mathtools}
\usepackage{enumitem}
\usepackage{hyperref}
\usepackage{cleveref}
\usepackage{thmtools}
\usepackage{booktabs}
\usepackage{array}
\usepackage[margin=1in]{geometry}
\usepackage{xcolor}
\usepackage{tcolorbox}

% Theorem environments
\theoremstyle{plain}
\newtheorem{theorem}{Theorem}[section]
\newtheorem{lemma}[theorem]{Lemma}
\newtheorem{proposition}[theorem]{Proposition}
\newtheorem{corollary}[theorem]{Corollary}

\theoremstyle{definition}
\newtheorem{definition}[theorem]{Definition}
\newtheorem{example}[theorem]{Example}
\newtheorem{remark}[theorem]{Remark}

\theoremstyle{remark}
\newtheorem*{notation}{Notation}
\newtheorem*{claim}{Claim}

% Custom commands
\newcommand{\E}{\mathbb{E}}
\newcommand{\R}{\mathbb{R}}
\newcommand{\N}{\mathbb{N}}
\newcommand{\C}{\mathcal{C}}
\newcommand{\M}{\mathcal{M}}
\newcommand{\Sol}{\mathcal{S}}
\newcommand{\Kol}{K}
\newcommand{\KP}{K_{\text{prefix}}}

% Title
\title{\textbf{Coherence-Rupture-Regeneration and Solomonoff Induction:\\
A Mathematical Comparison}}
\author{Mathematical Analysis}
\date{\today}

\begin{document}

\maketitle

\begin{abstract}
This document presents a mathematical analysis comparing the Coherence-Rupture-Regeneration (CRR) framework with Solomonoff induction. We establish correspondence theorems between the two frameworks and identify what each provides. The main findings are: (1) CRR coherence corresponds to accumulated conditional Kolmogorov complexity; (2) CRR rupture implements MDL model switching; (3) CRR regeneration weights correspond to Solomonoff's universal prior under specific parameter choices; (4) the frameworks are compatible, with CRR providing operational details that Solomonoff leaves unspecified. We also note limitations and open questions for both frameworks.
\end{abstract}

\tableofcontents
\newpage

%==============================================================================
\section{Introduction}
%==============================================================================

\subsection{Scope}

This document compares two frameworks for inductive inference:

\begin{enumerate}[label=(\roman*)]
    \item \textbf{Solomonoff Induction} (1964): A theoretical framework for sequence prediction based on algorithmic information theory and Kolmogorov complexity. It provides optimality guarantees but is incomputable.

    \item \textbf{Coherence-Rupture-Regeneration (CRR)}: A framework describing how bounded systems maintain identity through discontinuous change, grounded in Bayesian model comparison.
\end{enumerate}

We address three questions:
\begin{enumerate}
    \item What is the mathematical relationship between CRR and Solomonoff induction?
    \item Are the two frameworks compatible?
    \item What does each framework provide that the other does not?
\end{enumerate}

\subsection{Summary of Findings}

\begin{tcolorbox}[colback=gray!5,colframe=gray!75!black,title=Main Findings]
\textbf{What Solomonoff Induction Provides:}
\begin{itemize}
    \item The optimal prior: $P(h) = 2^{-K(h)}$
    \item The optimal posterior: $P(h|d) \propto P(d|h) \cdot 2^{-K(h)}$
    \item Proof of optimality (dominates all computable predictors)
\end{itemize}

\textbf{What Solomonoff Does Not Provide:}
\begin{itemize}
    \item When to act on accumulated evidence
    \item A computable implementation
    \item The mechanism of model change
    \item A tunable parameter for different contexts
    \item Discrete update structure
\end{itemize}

\textbf{What CRR Adds:}
\begin{itemize}
    \item Decision rule: act when $\C \geq \Omega$
    \item Computable approximation via temporal coherence accumulation
    \item Regeneration operator specifying how models are reconstructed
    \item Tunable rigidity parameter $\Omega$
    \item Discrete rupture moments
\end{itemize}
\end{tcolorbox}

\subsection{The Three-Framework Synthesis}

CRR can be understood in relation to two other frameworks:

\begin{center}
\begin{tabular}{@{}lll@{}}
\toprule
\textbf{Framework} & \textbf{Specifies} & \textbf{Mathematical Form} \\
\midrule
Solomonoff & What is optimal & $P(h) = 2^{-K(h)}$ \\
Free Energy Principle & What to minimize & $F = \E[\log q] - \E[\log p]$ \\
CRR & When and how & $\C \to \delta \to R$ with threshold $\Omega$ \\
\bottomrule
\end{tabular}
\end{center}

These frameworks address different aspects of inference and are not mutually exclusive.

%==============================================================================
\section{Preliminaries}
%==============================================================================

\subsection{Algorithmic Information Theory}

\begin{definition}[Kolmogorov Complexity]
The \emph{Kolmogorov complexity} of a string $x$ with respect to universal machine $U$ is:
\begin{equation}
K_U(x) = \min\{|p| : U(p) = x\}
\end{equation}
where $|p|$ denotes the length of program $p$ in bits.
\end{definition}

\begin{definition}[Conditional Kolmogorov Complexity]
The complexity of $x$ given $y$ is:
\begin{equation}
K(x|y) = \min\{|p| : U(p, y) = x\}
\end{equation}
\end{definition}

\begin{theorem}[Invariance Theorem]
For any two universal machines $U_1, U_2$, there exists a constant $c$ such that for all $x$:
\begin{equation}
|K_{U_1}(x) - K_{U_2}(x)| \leq c
\end{equation}
\end{theorem}

\subsection{Solomonoff Induction}

\begin{definition}[Solomonoff Prior]
The \emph{Solomonoff prior} over binary strings is:
\begin{equation}
\Sol(x) = \sum_{p: U(p) = x^*} 2^{-|p|}
\end{equation}
where $x^*$ denotes strings beginning with $x$.
\end{definition}

\begin{definition}[Universal Prediction]
Given observed sequence $x_{1:n}$, the Solomonoff predictor assigns:
\begin{equation}
P_{\Sol}(x_{n+1} | x_{1:n}) = \frac{\Sol(x_{1:n+1})}{\Sol(x_{1:n})}
\end{equation}
\end{definition}

\begin{theorem}[Solomonoff Convergence]
Let $\mu$ be any computable measure. The expected total squared prediction error is bounded:
\begin{equation}
\E_\mu\left[\sum_{n=1}^{\infty} (P_{\Sol}(x_{n+1}|x_{1:n}) - \mu(x_{n+1}|x_{1:n}))^2\right] \leq K(\mu) \ln 2
\end{equation}
\end{theorem}

\begin{remark}[Incomputability]
Solomonoff induction is incomputable because computing $K(x)$ requires solving the halting problem.
\end{remark}

\subsection{CRR Framework}

\begin{definition}[CRR System]
A \emph{CRR system} is a tuple $(\M, Y, \Pi, \Omega, \C, R)$ where:
\begin{itemize}
    \item $\M = \{m, m', \ldots\}$ is a set of generative models
    \item $Y = \R^d$ is the observation space
    \item $\Pi: \M \to \text{PD}(d)$ assigns precision matrices
    \item $\Omega: \M \times \M \to \R$ is the rigidity function
    \item $\C: \M \times \N \to \R_+$ is the coherence accumulator
    \item $R$ is the regeneration operator
\end{itemize}
\end{definition}

\begin{definition}[CRR Operators]
The three CRR operators are:

\textbf{Coherence:}
\begin{equation}
\C_m(n) = \frac{1}{2}\sum_{i=1}^{n} (y_i - g_m(\mu_i))^\top \Pi_m (y_i - g_m(\mu_i))
\end{equation}

\textbf{Rupture:} Occurs when $\C_m(n) - \C_{m'}(n) > \Omega_{m,m'}$

\textbf{Regeneration:}
\begin{equation}
R[\phi](t) = \frac{1}{Z}\int_0^t \phi(\tau) \cdot \exp\left(\frac{\C(\tau)}{\Omega}\right) \cdot \Theta(t-\tau) \, d\tau
\end{equation}
\end{definition}

%==============================================================================
\section{Correspondence Theorems}
%==============================================================================

\subsection{Coherence and Kolmogorov Complexity}

\begin{theorem}[Coherence-Complexity Correspondence]\label{thm:coherence-complexity}
Let $y_{1:n}$ be an observation sequence and $m$ a computable generative model. Define the \emph{algorithmic coherence}:
\begin{equation}
\C^{\text{alg}}_m(n) = \sum_{i=1}^{n} K(y_i | y_{<i}, m)
\end{equation}
Then for CRR coherence under Gaussian observations:
\begin{equation}
\C_m(n) = \C^{\text{alg}}_m(n) + O(n)
\end{equation}
where the $O(n)$ term accounts for the model's encoding overhead.
\end{theorem}

\begin{proof}
The proof proceeds in three steps.

\textbf{Step 1:} By Levin's coding theorem, for any computable distribution $P$:
\begin{equation}
-\log P(x) = K(x) + O(K(P))
\end{equation}

\textbf{Step 2:} CRR coherence under Gaussian observations equals negative log-likelihood:
\begin{equation}
\C_m(n) = -\log p(y_{1:n} | m) + \frac{n}{2}\log\det(2\pi\Sigma_m)
\end{equation}

\textbf{Step 3:} Combining these:
\begin{align}
\C_m(n) &= \sum_{i=1}^{n} [-\log p(y_i | y_{<i}, m)] + O(n) \\
&= \sum_{i=1}^{n} [K(y_i | y_{<i}, m) + O(1)] + O(n) \\
&= \C^{\text{alg}}_m(n) + O(n)
\end{align}
\end{proof}

\begin{remark}[Temporal Survival as Complexity Proxy]
A key operational insight: patterns that survive repeated validation under CRR \emph{behave as if} they have low Kolmogorov complexity, without ever computing $K$ directly. Low coherence accumulation over time indicates the model compresses observations well. This provides a computable proxy for the incomputable complexity measure.
\end{remark}

\subsection{Rupture and Minimum Description Length}

\begin{theorem}[Rupture as MDL Model Switching]\label{thm:rupture-mdl}
The CRR rupture condition is equivalent to the MDL model switching criterion:
\begin{equation}
\text{Rupture } m \to m' \iff \C_m(n) > \C_{m'}(n) + \Omega_{m,m'}
\end{equation}
where $\Omega_{m,m'} = K(m') - K(m) + K(\text{switch})$ is the description length overhead.
\end{theorem}

\begin{proof}
The MDL principle selects the model minimizing $K(m) + K(y_{1:n} | m)$.

Model $m'$ is preferred over $m$ when:
\begin{equation}
K(m') + K(y_{1:n} | m') < K(m) + K(y_{1:n} | m)
\end{equation}

Rearranging and substituting CRR variables via Theorem \ref{thm:coherence-complexity}:
\begin{equation}
\C_m(n) - \C_{m'}(n) > K(m') - K(m) = \Omega_{m,m'}
\end{equation}
\end{proof}

\subsection{Regeneration and Universal Prior}

\begin{theorem}[Regeneration-Prior Correspondence]\label{thm:regeneration-solomonoff}
The CRR regeneration weighting $\exp(\C(\tau)/\Omega)$ corresponds to Solomonoff's universal prior under specific parameter identification.
\end{theorem}

\begin{proof}
The Solomonoff prior assigns $P(\phi) \propto 2^{-K(\phi)}$.

The CRR regeneration weights by $w(\tau) \propto \exp(\C(\tau)/\Omega)$.

Since $\C(\tau) \approx -\log p(y_{1:\tau} | \phi) \approx K(y_{1:\tau} | \phi)$ by Levin's theorem:
\begin{equation}
w(\phi) \propto \exp\left(\frac{-K(y | \phi)}{\Omega}\right) = 2^{-K(y|\phi)/(\Omega \ln 2)}
\end{equation}

\textbf{Parameter identification:} For exact correspondence with Solomonoff, we require:
\begin{equation}
\Omega = \frac{1}{\ln 2} \approx 1.443
\end{equation}

For other values of $\Omega$, the correspondence holds up to a scaling of the exponent. With $\Omega = 1/\pi \approx 0.318$ (as conjectured in CRR), the exponent is scaled by a factor of approximately $4.5$.
\end{proof}

\begin{remark}[Scaling Implications]
The choice of $\Omega$ affects how aggressively complexity is weighted:
\begin{itemize}
    \item $\Omega = 1/\ln 2$: Exact Solomonoff correspondence
    \item $\Omega < 1/\ln 2$: More aggressive complexity penalization
    \item $\Omega > 1/\ln 2$: More lenient complexity penalization
\end{itemize}
\end{remark}

%==============================================================================
\section{What Solomonoff Provides}
%==============================================================================

Solomonoff induction provides:

\begin{enumerate}
    \item \textbf{The optimal prior:} $P(h) = 2^{-K(h)}$ assigns higher probability to simpler hypotheses in an objective, machine-independent way (up to constants).

    \item \textbf{The optimal posterior:} $P(h|d) \propto P(d|h) \cdot 2^{-K(h)}$ follows from Bayes' theorem.

    \item \textbf{Optimality proof:} Solomonoff's convergence theorem shows this predictor dominates all computable predictors—total prediction error is bounded by the complexity of the true distribution.
\end{enumerate}

%==============================================================================
\section{What Solomonoff Does Not Provide}
%==============================================================================

\subsection{When to Update}

Solomonoff induction is a ratio of probabilities. It does not specify when an agent should act on accumulated evidence or commit to a model. The posterior weights change continuously, but there is no threshold or decision rule.

\textbf{CRR adds:} The rupture condition $\C_m - \C_{m'} > \Omega$ provides an explicit decision rule. The rupture operator $\delta$ marks discrete decision moments.

\subsection{Computability}

$K(h)$ is uncomputable—there is no algorithm to compute Kolmogorov complexity for arbitrary strings.

\textbf{CRR adds:} Temporal coherence accumulation as a proxy. Models that survive repeated validation (low $\C$ accumulation) behave as if they have low $K$, without computing $K$ directly. This is the operational insight—complexity is approximated through temporal survival.

\subsection{Mechanism of Model Change}

Solomonoff gives posteriors but not how one model replaces another in a physical system.

\textbf{CRR adds:} The regeneration operator $R = \int\phi \cdot \exp(\C/\Omega) \cdot \Theta \, d\tau$ specifies that new models are reconstructed from memory weighted by coherence.

\subsection{Tunable Parameter}

Solomonoff's prior is fixed by the choice of universal machine. There is no parameter to adjust for different contexts.

\textbf{CRR adds:} The rigidity parameter $\Omega$, which can potentially vary across:
\begin{itemize}
    \item Individuals (cognitive style)
    \item Domains (expertise level)
    \item States (arousal, attention)
    \item Timescales (different $\Omega$ for different temporal scales)
\end{itemize}

\begin{remark}[Caution on $\Omega$ Variability]
The variability of $\Omega$ is both a feature and a potential concern. If $\Omega$ can be freely adjusted, it risks becoming a free parameter that can fit any data. The CRR documentation partially addresses this with Kac's lemma ($\Omega = 1/\mu(A)$), which derives $\Omega$ from the measure of the coherent region. Whether a universal value exists remains an open question.
\end{remark}

\subsection{Discrete Update Structure}

Solomonoff is typically presented as continuous updating—posterior weights change smoothly with each observation.

\textbf{CRR adds:} Updates occur at discrete rupture moments—the $\delta$ that marks threshold crossings. This connects to process philosophy (Whitehead's ``actual occasions'') and stopping time theory in probability.

%==============================================================================
\section{Compatibility Analysis}
%==============================================================================

\begin{theorem}[Compatibility]\label{thm:compatibility}
CRR and Solomonoff induction are mathematically compatible. CRR operationalizes Solomonoff for finite agents by replacing:
\begin{center}
\begin{tabular}{@{}ll@{}}
\toprule
\textbf{Solomonoff (incomputable)} & \textbf{CRR (computable proxy)} \\
\midrule
Kolmogorov complexity $K$ & Accumulated coherence $\C$ \\
Continuous updating & Discrete rupture at $\C = \Omega$ \\
Abstract posterior & Concrete regeneration dynamics \\
Fixed prior & Tunable rigidity $\Omega$ \\
\bottomrule
\end{tabular}
\end{center}
\end{theorem}

\begin{proof}
Compatibility follows from Theorems \ref{thm:coherence-complexity}, \ref{thm:rupture-mdl}, and \ref{thm:regeneration-solomonoff}, which establish correspondences between:
\begin{itemize}
    \item Coherence and conditional Kolmogorov complexity
    \item Rupture and MDL model switching
    \item Regeneration weights and the universal prior
\end{itemize}

The discretization of continuous updating into threshold-triggered ruptures is analogous to the relationship between Bayesian updating and the Sequential Probability Ratio Test (SPRT). Both implement Bayesian inference; SPRT adds stopping rules.
\end{proof}

\begin{remark}[One-Sentence Summary]
CRR describes how bounded agents can approximate theoretically optimal inference through temporal dynamics and threshold-triggered discrete updates.
\end{remark}

%==============================================================================
\section{Comparison Tables}
%==============================================================================

\begin{table}[h!]
\centering
\caption{What Each Framework Provides}
\label{tab:provides}
\begin{tabular}{@{}p{5cm}cc@{}}
\toprule
\textbf{Feature} & \textbf{Solomonoff} & \textbf{CRR} \\
\midrule
Optimal prior & Yes & No (uses arbitrary $p(m)$) \\
Optimal prediction guarantee & Yes & No \\
Decision rule (when to act) & No & Yes ($\C \geq \Omega$) \\
Computable & No & Yes (for finite $\M$) \\
Model change mechanism & No & Yes (regeneration) \\
Tunable parameter & No & Yes ($\Omega$) \\
Multiscale structure & No & Yes \\
Discrete update structure & No & Yes (rupture) \\
\bottomrule
\end{tabular}
\end{table}

\begin{table}[h!]
\centering
\caption{Mathematical Correspondences}
\label{tab:correspondence}
\begin{tabular}{@{}ll@{}}
\toprule
\textbf{Solomonoff Induction} & \textbf{CRR Framework} \\
\midrule
$K(y|m)$ & $\C_m(n) + O(n)$ \\
$2^{-K(m)}$ & $p(m)$; $\Omega = \log(p(m)/p(m'))$ \\
Posterior update & Rupture condition \\
$2^{-K(\phi)}$ weighting & $\exp(\C/\Omega)$ weighting \\
Incomputable & Computable \\
\bottomrule
\end{tabular}
\end{table}

%==============================================================================
\section{Limitations and Open Questions}
%==============================================================================

\subsection{Limitations of CRR Relative to Solomonoff}

\begin{enumerate}
    \item \textbf{Model class restriction:} CRR requires specifying $\M$ in advance; Solomonoff considers all computable hypotheses.

    \item \textbf{No universal convergence:} Solomonoff guarantees convergence to any computable truth. CRR guarantees optimal switching within $\M$, but may fail if the truth is outside $\M$.

    \item \textbf{Discretization error:} CRR's discrete ruptures may miss continuous optimality of Solomonoff's mixture.

    \item \textbf{Parameter dependence:} CRR depends on $\Omega$, which may be difficult to determine.
\end{enumerate}

\subsection{Limitations of Solomonoff Relative to CRR}

\begin{enumerate}
    \item \textbf{Incomputability:} Solomonoff cannot be implemented; CRR can.

    \item \textbf{No decision rule:} Solomonoff provides probabilities but not when to act.

    \item \textbf{No temporal structure:} Solomonoff does not address discrete moments of change.

    \item \textbf{No multiscale analysis:} Solomonoff operates at a single scale.
\end{enumerate}

\subsection{Open Questions}

\begin{enumerate}
    \item \textbf{Universal $\Omega$:} Can $\Omega = 1/\pi$ be derived from first principles? Neither framework provides such a derivation.

    \item \textbf{Convergence rate:} How does CRR's approximation error scale as $|\M| \to \infty$?

    \item \textbf{Continuous limit:} Does CRR with $\Omega \to 0$ converge to Solomonoff's mixture?

    \item \textbf{Empirical validation:} The claimed correspondence requires empirical testing across domains.
\end{enumerate}

%==============================================================================
\section{Conclusion}
%==============================================================================

This analysis establishes the mathematical relationship between CRR and Solomonoff induction:

\begin{enumerate}
    \item \textbf{Compatibility:} The frameworks are compatible. CRR coherence corresponds to accumulated Kolmogorov complexity, rupture to MDL model switching, and regeneration to the universal prior (under parameter identification).

    \item \textbf{Complementarity:} Solomonoff specifies \emph{what} is optimal; CRR specifies \emph{when} and \emph{how}. Neither is a substitute for the other.

    \item \textbf{Operationalization:} CRR provides a computable approximation to Solomonoff by using temporal coherence accumulation as a proxy for algorithmic complexity.
\end{enumerate}

The relationship can be summarized as: \emph{CRR is to Solomonoff induction as SPRT is to Bayesian updating}—both pairs share theoretical foundations, but the former adds actionable stopping rules.

%==============================================================================
\begin{thebibliography}{99}

\bibitem{solomonoff1964}
Solomonoff, R.J. (1964). A formal theory of inductive inference. \textit{Information and Control}, 7(1), 1-22.

\bibitem{kolmogorov1965}
Kolmogorov, A.N. (1965). Three approaches to the quantitative definition of information. \textit{Problems of Information Transmission}, 1(1), 1-7.

\bibitem{levin1974}
Levin, L.A. (1974). Laws of information conservation and aspects of the foundation of probability theory. \textit{Problems of Information Transmission}, 10(3), 206-210.

\bibitem{rissanen1978}
Rissanen, J. (1978). Modeling by shortest data description. \textit{Automatica}, 14(5), 465-471.

\bibitem{wald1947}
Wald, A. (1947). \textit{Sequential Analysis}. John Wiley \& Sons.

\bibitem{hutter2005}
Hutter, M. (2005). \textit{Universal Artificial Intelligence}. Springer.

\bibitem{grunwald2007}
Gr\"{u}nwald, P.D. (2007). \textit{The Minimum Description Length Principle}. MIT Press.

\bibitem{liandvitanyi2008}
Li, M. \& Vit\'{a}nyi, P. (2008). \textit{An Introduction to Kolmogorov Complexity and Its Applications}. Springer.

\end{thebibliography}

%==============================================================================
\appendix
\section{Technical Details}
%==============================================================================

\subsection{Levin's Coding Theorem}

\begin{theorem}[Levin]
For any computable probability distribution $P$ and string $x$:
\begin{equation}
-\log P(x) = K(x) + O(K(P))
\end{equation}
\end{theorem}

\subsection{Chain Rule for Kolmogorov Complexity}

\begin{theorem}[Chain Rule]
For strings $x, y$:
\begin{equation}
K(x, y) = K(x) + K(y|x^*) + O(\log K(x,y))
\end{equation}
where $x^*$ is the shortest program for $x$.
\end{theorem}

\subsection{SPRT Optimality}

\begin{theorem}[Wald-Wolfowitz]
Among all sequential tests with type I error $\leq \alpha$ and type II error $\leq \beta$, the SPRT minimizes the expected sample size under both hypotheses.
\end{theorem}

CRR rupture is equivalent to SPRT when the log-likelihood ratio equals the coherence difference:
\begin{equation}
\Lambda_n = \log\frac{p(y_{1:n}|m')}{p(y_{1:n}|m)} = \C_m(n) - \C_{m'}(n)
\end{equation}

\end{document}
